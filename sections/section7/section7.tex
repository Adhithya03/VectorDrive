\section{CONCLUSION}

\subsection{GENERAL}

In conclusion, this thesis successfully demonstrates the design and implementation of an DSP based Sensorless Field Oriented Control of Induction Motor and proves that the smooth and precise control of speed and torque of vector control is unmatched by any other control strategies. The only input to the controller from the user for a given motor is the set speed and load applied to the shaft, while the control takes care of the rest and achieves the set speed in a quick and smooth manner driving the rated load even at low speeds.

The system effectively regulates key motor variables, achieving desired performance levels. Both the magnetizing current (Id) and motor speed accurately track their respective reference values, demonstrating precise control. The Id reference and feedback values converge to 0.2 pu, ensuring proper motor magnetization. Similarly, the motor speed closely follows its reference value of 0.5 pu. Furthermore, the load angle, a critical parameter for torque generation, stabilizes around 85-90 degrees. This indicates operation near the maximum torque per ampere condition, maximizing the motor's efficiency. The control system utilizes a carrier frequency of 15 kHz for pulse width modulation.  It's important to note that the rated frequency for achieving the rated speed of the motor is 50 Hz. The control system dynamically adjusts the output frequency based on the desired speed reference, ensuring optimal performance across the operating range.


These results validate the FOC system's ability to achieve stable, controlled, and efficient operation of the AC induction motor under the simulated conditions.


While the project faced challenges that prevented the full realization of a closed-loop FOC system, the accomplishments in simulation, control algorithm design, and PCB design provide a valuable basis for future work. Addressing the remaining hardware issues and implementing the closed-loop control will enable the validation of the designed system and its potential application in various motor control scenarios.


\subsection{SCOPE FOR FUTURE WORK}

Future work fo this thesis can focus on integrating position sensors and exploring advanced control techniques for further optimization. Though going sensorless reduces the hardware complexity and overall cost, it proportionally increases the software design complexity and possible reduction in position estimation accuracy due to errors in measurement of motor parameters. Using a rotary encoder with enough resolution will remove the possibility of inaccurate position estimations. Other control strategies include Direct Torque Control, Model Predictive Control and Indirect Field Oriented Control. Though Direct FOC is the overall best vector control algorithm, other mentioned control strategies have their own application specific advantages that can be further analysed and proper comparsion between all control strategies can be recorded.


\newpage


\begin{center}
      {\textbf{REFERENCES}}
\end{center}

\begin{enumerate}
      \item Chen, Wen Zhuo, et al. “Simulation of Permanent Magnet Synchronous Motor Field Oriented Vector Control System.” Applied Mechanics and Materials, vol. 672–674, Trans Tech Publications, Ltd., Oct. 2014, pp. 1234–1237.


            \vspace{5mm} % add 10 millimeters of space

      \item D Arun Dominic, Thanga Raj Chelliah,
            Analysis of field-oriented controlled induction motor drives under sensor faults and an overview of sensorless schemes,
            ISA Transactions,
            Volume 53, Issue 5,
            2014,
            Pages 1680-1694,
            ISSN 0019-0578.

            \vspace{5mm} % add another 10 millimeters of space

      \item D. Xu, B. Wang, G. Zhang, G. Wang and Y. Yu, "A review of sensorless control methods for AC motor drives," in CES Transactions on Electrical Machines and Systems, vol. 2, no. 1, pp. 104-115, March 2018.


            \vspace{5mm} % add another 10 millimeters of space

      \item  E. S. Tez, "A simple understanding of field-orientation for AC motor control," IEE Colloquium on Vector Control and Direct Torque Control of Induction Motors, London, UK, 1995, pp. 3/1 -3/4.

            \vspace{5mm} % add another 10 millimeters of space


      \item  Yusivar, N. Hidayat, R. Gunawan and A. Halim, "Implementation of field oriented control for permanent magnet synchronous motor," 2014 International Conference on Electrical Engineering and Computer Science (ICEECS), Kuta, Bali, Indonesia, 2014, pp. 359-362.
            % Rupprecht Gabriel.F.

\end{enumerate}
\subsection*{WEB REFERENCES AND VIDEOS}


\begin{enumerate}
      \item Teaching Old Motors New Tricks - An educational resource by Texas Instruments, outlining modern methods for enhancing older motor systems.\\
            View at: \url{https://www.youtube.com/watch?v=fpTvZlnrsP0}

      \item Electrical Machines and Drives by Theodore Wildi

      \item Motor Control Series by Matlab - Provides a structured learning path for motor control systems.\\
            Available at: \url{https://www.youtube.com/watch?v=gNpoTPzEkco}

      \item Hardware Design for Variable Frequency Drive(VFD) by Matan Pazi.\\
            Watch at: \url{https://youtu.be/wUGCEtSXV1I?si=l1Pyao91pGLhClpd}

      \item Park Transformation by Jantzen Lee.\\
            View at: \url{https://youtu.be/mbJOxqxLkLE?si=CPC0R3QVOjHkjJIL}

      \item Computing Load Angle of Induction motor from Matlab.\\
            Available at: \url{https://www.mathworks.com/help/mcb/ref/fluxobserver.html}

      \item C2000 Enhanced Pulse Width Modulator (ePWM):\\
            View at: \url{https://www.ti.com/video/series/C2000-enhanced-pulse-width-modulator.html}

      \item PCB design with Ultiboard and Multisim.\\
            Watch at: \url{https://www.youtube.com/watch?v=DtPCK3qGakM&list=PLVg5xjDHQldd2SjGsXRB4atrrWZ9rLCe_}

      \item TMS320F2837xD Dual-Core Real-Time Microcontrollers Technical Reference Manual.
      
      \item AN-9121 Smart Power Module, Motion 600 V SPM 2 Series User's Guide.
      
      \item FSAM20SH60A Motion SPM 2 Series Datasheet January 2014.
      
      \item  Sensorless Field Oriented Control of 3-Phase Permanent Magnet Synchronous Motors With CLA: Application Report SPRABQ5-July 2013.

      \item DIP SPM® Application Note (2012-07-09) Application Note AN-9043 Smart Power Module Motion SPM® Device in DIP (SPM2 V1) User`s Guide
 
\end{enumerate}

