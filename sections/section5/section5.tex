\subsection{Induction Motor Parameter Estimation using No-load and Blocked Rotor Test}

% Circuit diagram at no and blocked from krishnan

\subsubsection{Cold Test}

\subsubsection{No-load Test}

% hardware setup for no-load test

\begin{figure}
	\centering
	\includegraphics[width=4in]{sections/section5/images/ParamEstim/SetupNoload.jpg}
	\caption{No-load test setup}
	\label{fig:no_load_test_setup}
\end{figure}

The no-load setup has 1Hp induction motor connected to variable frequency drive (VFD) to supply no-load losses like friction and windage losses of 0.25Hp motor which is connected to the power analyzer.


% Fluke 434 analyzer image voltage amps hertz is shown in the figure

\begin{figure}[H]
	\centering
	\includegraphics[width=4in]{sections/section5/images/ParamEstim/FlukeVoltAmpHertz.png}
	\caption{Fluke 434 power analyzer}
	\label{fig:fluke434}
\end{figure}

% From R Krishnan book circuit equivalent figure

\begin{figure}[H]
	\centering
	\includegraphics[width=4in]{sections/section5/images/ParamEstim/noloadCircuitKrish.png}
	\caption{No-load test circuit}
	\label{fig:no_load_test}
\end{figure}

The no-load power factor is given by:
$$\cos \phi_0 = \frac{P_i}{V_\text{as}I_0}$$

The magnetizing current is calculated as:
$$I_m = I_0 \sin \phi_0$$

The core-loss current is given by:
$$I_c = I_0 \cos \phi_0$$

The magnetizing inductance is computed from:
$$L_m = \frac{V_\text{as}}{2\pi f_\text{i}I_m}$$

The core-loss resistance is given by:
$$R_c = \frac{V_\text{as}}{I_c}$$


\subsubsection{Blocked Test}



The short-circuit power factor obtained from the equivalent circuit is:
$$\cos \phi_\text{sc} = \frac{P_\text{sc}}{V_\text{sc}I_\text{sc}}$$

The short-circuit impedance is given by:
$$Z_\text{sc} = \frac{V_\text{sc}}{I_\text{sc}}$$

From which the rotor resistance and total leakage reactance are computed as:
$$R_r = Z_\text{sc} \cos \phi_\text{sc} - R_s$$
$$X_\text{eq} = Z_\text{sc} \sin \phi_\text{sc}$$

where the total leakage reactance per phase, $X_\text{eq}$, is the sum of the stator and referred-rotor leakage reactances, given as:
$$X_\text{eq} = X_\text{ls} + X_\text{lr}$$



Based on the above test and calculations, the equivalent circuit parameters are computed as follows:

% Alighn left

\begin{flalign*}
	&L_m = 1.01, R_c = 1555.85                  &\\
	&R_r = 5.02, X_{ls} = 50.47, X_{lr} = 50.47 &
\end{flalign*}


\newpage