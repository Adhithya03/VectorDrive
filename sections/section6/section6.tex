\subsection{Space Vector Pulse Width Modulation}

Space Vector PWM has several advantages over Sine PWM

\begin{itemize}
    \item Higher voltage utilization: SVPWM can utilize up to 15\% more DC bus voltage compared to SPWM. This means for the same DC supply voltage, an inverter with SVPWM can provide a higher output voltage.

    \item Better harmonic performance: SVPWM results in lower total harmonic distortion (THD) compared to SPWM. This leads to a better quality of the output voltage and current waveforms, which is particularly important in applications like drives where harmonics can cause heating and torque pulsations.

    \item Reduced switching losses: SVPWM requires fewer switching operations for the inverter switches compared to SPWM. This results in lower switching losses, leading to higher efficiency and reduced heating of the inverter switches.

    \item Improved dynamic response: The space vector representation used in SVPWM allows for a more precise control of the output voltage vector, leading to an improved dynamic response. This is particularly beneficial in applications like motor drives where a fast dynamic response is required.

    \item Vector control capability: SVPWM allows for vector control of the output voltage, which is not possible with SPWM. This enables more complex control strategies, such as field-oriented control (FOC), which can provide better performance in applications like motor drives.

    \item Flexibility: SVPWM allows for flexible control of the output voltage magnitude and frequency, as well as the phase relationship between the output voltage and current. This flexibility makes it suitable for a wide range of applications. 
\end{itemize}

\subsubsection{Generation of Space Vector PWM with C2000 microcontroller}

To generate space vector PWM wave for the switches C2000 series microcontroller offers a hardware level module called ePWM or enhanced PWM module. It enables to generate PWM waves with high flexibility.

To generate symmetrical waveform, the ePWM's internal timer is configured in up-down count mode.

\section*{PWM Frequency Calculation}

\subsection*{Variable Definitions}

\renewcommand{\arraystretch}{1.5}
\begin{tabular}{|>{\bfseries}l|l|}
    \hline
    Symbol & Description \\ \hline
    $F_{PWM}$ and $T_{PWM}$ & Frequency of PWM (Hz) and Time period of PWM (seconds) \\ \hline
    $TBCLK$ & Time base clock (Hz)\\ \hline
    $T_{TBCLK}$ & Time period of time base clock (in seconds) \\ \hline
    $TBPRD$ & Timer period (in clock cycles) \\ \hline
    $EPWMCLK$ & ePWM module clock (in Hz)\\ \hline
    $HSPCLKDIV$ & High speed clock divider\\ \hline
    $CLKDIV$ & Clock divider \\ \hline
\end{tabular}


The period of the PWM signal can be calculated using the formula:
\[
T_{\text{PWM}} = 2 \times TBPRD \times T_{\text{TBCLK}}
\]
where \( TBPRD \) is the time base period.

\subsection{PWM Frequency (\( F_{\text{PWM}} \))}
The frequency of the PWM signal is defined as the inverse of the PWM period:
\[
F_{\text{PWM}} = \frac{1}{T_{\text{PWM}}}
\]

\subsection{Time Base Clock (\( T_{\text{TBCLK}} \))}
The time base clock is given by:
\[
T_{\text{TBCLK}} = \frac{\text{EPWMCLK}}{\text{HSPCLKDIV} \times \text{CLKDIV}}
\]
\begin{itemize}
    \item \( \text{EPWMCLK} \) is the clock frequency dedicated to the PWM module.
    \item \( \text{HSPCLKDIV} \) and \( \text{CLKDIV} \) are the dividers for the high-speed PWM clock.
\end{itemize}

According to the FSAM20SH60A datasheet, a 15 kHz carrier wave is recommended for optimal performance. The dividers \( \text{HSPCLKDIV} \) and \( \text{CLKDIV} \) are both set to 1. Given that the EPWMCLK is derived from the system clock (SYSCLK) which operates at 200 MHz, the time base clock can be calculated as follows:
\[
T_{\text{TBCLK}} = \frac{200 \times 10^6}{1 \times 1} = 200 \times 10^6 \text{ Hz}
\]

\[
T_{\text{TBCLK}}

\]

Figure \ref{fig:svpwm_block} below shows the ePWM block in Simulink, which is used to simulate the PWM control scheme.

\begin{figure}[ht]
    \centering
    \includegraphics[width=1.8in]{sections/section6/images/SVPWM/ePWMBlock.png}
    \caption{ePWM block in Simulink}
    \label{fig:svpwm_block}
\end{figure}

\newpage